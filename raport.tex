\documentclass{article}

\usepackage[utf8]{inputenc}
\usepackage[T1]{fontenc}
\usepackage{titlesec}
\usepackage{tocloft}
\usepackage{amsmath}
\usepackage{geometry}

\renewcommand{\tablename}{Przykład}

\geometry{a4paper, margin=1in}

\title{Metoda Jacobiego rozwiązywania układów równań liniowych - Raport}
\author{Mateusz Karandys}
\date{\today}

\begin{document}
	
	\maketitle
	
	\section{Wstęp}
	Zajmiemy się problemem iteracyjnego wyznaczania rozwiązań układu równań $Ax=b$ dla szczególnej postaci macierzy $A$. Będziemy rozważać macierz postaci:
	\[
	A = 
	\begin{bmatrix}
			C & S \\
			-S & C \\
	\end{bmatrix}
	\]
	gdzie $A$ jest wymiaru $2p \times 2p$, $C = diag(c_1, c_2, \ldots, c_p)$ przy czym $detC \ne 0$, $S = diag(s_1, s_2, \ldots, s_p)$ i $\forall i \in \{1, 2, \ldots, p\}$ $c_i^2 + s_i^2 = 1$.
	
	\section{Metoda Jacobiego}
	% Treść opisu metody
		
	Zapiszmy macierz $A$ w postaci sumy trzech macierzy $L$, $D$, $U$, gdzie $L$ jest macierzą, której niezerowe elementy znajdują się pod główną diagonalą, $D$ jest macierzą diagonalną, $U$ jest macierzą o niezerowych elementach znajdujących się powyżej głównej diagonali.
	
	\[
	A = L + D + U =
	\]\[
	\begin{bmatrix}
		0 & 0 & \cdots & 0 & 0 \\
		a_{21} & 0  & \cdots & 0 & 0 \\
		a_{31} & a_{32}  & \cdots & 0 & 0 \\
		\vdots & \vdots & \ddots & \vdots & \vdots \\
		a_{n1} & a_{n2} & \cdots & a_{n,n-1} & 0
	\end{bmatrix}
	+
	\begin{bmatrix}
		a_{11} & 0 & 0 & \cdots & 0 \\
		0 & a_{22} & 0 & \cdots & 0 \\
		0 & 0 & a_{33} & \cdots & 0 \\
		\vdots & \vdots & \vdots & \ddots & \vdots \\
		0 & 0 & 0 & \cdots & a_{nn}
	\end{bmatrix}
	+
	\begin{bmatrix}
		0 & a_{12} & a_{13} & \cdots & a_{1n} \\
		0 & 0 & a_{23} & \cdots & a_{2n} \\
		\vdots & \vdots & \vdots & \ddots & \vdots \\
		0 & 0 & 0 & \cdots & a_{n-1,n} \\
		0 & 0 & 0 & \cdots & 0
	\end{bmatrix}
	\]
	Po podstawieniu tak zapisanej macierzy $A$ do układu $Ax = b$ otrzymujemy 
	\[
	(L + D + U)x = b 
	\]
	co można zapisać jako 
	
	\begin{gather*}
	Dx + (L + U)x = b \\
	Dx = -(L + U)x + b \\
	x = -D^{-1}(L + U)x + D^{-1}b
	\end{gather*}
	Po dopisaniu numeru iteracji mamy
	\[
	x^{(k+1)} = -D^{-1}(L + U)x^{(k)} + D^{-1}b,
	\quad k = 0, 1, \ldots
	\]
	
	\section{Implementacja}
	Implementację przygotowaliśmy w środowisku MATLAB. Zaimplementowaliśmy funkcję wyznaczającą rozwiązanie zadanego układu przy użyciu metody Jacobiego. Wybraliśmy przybliżenie początkowe $x = [0, \ldots, 0]^T$. Jako warunek stopu wybraliśmy nierówność $||x_k - x_{k-1}||_2 \le \varepsilon$, gdzie $\varepsilon$ jest podawany przez użytkownika. Dodatkowo narzuciliśmy ograniczenie na liczbę iteracji, aby program nie wykonywał się w nieskończoność w przypadku braku zbieżności. Ograniczeniem jest $100000$. Wyniki porównywaliśmy z wbudowaną funkcją $linsolve$. Do porównania wyników przygotowaliśmy pomocniczą funkcję zwracającą wyniki w postaci dwóch tabel.
	
	\section{Obsługa Skryptu}
	W skrypcie $przyklady.m$ znajdują się badane przypadki metody. W celu wykonania przykładu należy odkomentować linie kodu mu odpowiadające, a następnie uruchomić skrypt. Poniżej znajduje się specyfikacja zaimplementowanych funkcji.
	
	\begin{enumerate}
	\item $jacobi.m$ - $jacobi(c, s, b, stop)$ Wyznacza rozwiązanie układu równań $Ax = b$ metodą Jacobiego. $c, s$ - poziome wektory tej samej długości - wyznaczające macierz układu, $b$ - kolumna	wyrazów wolnych układu, $stop$ - wartość toleracji błedu. Funkcja zwraca dwie wartości: $x$ - rozwiązanie układu,
	$cnt$ - liczbę wykonanych iteracji (max $100000$)
	
	\item $testSolve.m$ - $testSolve(c, s, b)$ Wyznacza rozwiązanie układu równań $Ax = b$. Parametry $c$, $s$ oraz $b$ są takie same jak w funkcji wyżej.
	
	\item $porownaj.m$ - $porownaj(c, s, b, d)$ Zwraca dwie tabele, jedną zawierającą zestawienie wektorów $c$ i $s$, drugą z zestawieniem wyników powyższych funkcji. Parametry $c$, $s$, $b$ jak wyżej. $d$ jest warunkiem stopu (odpowiednik $stop$ w funkcji $jacobi$).
	\end{enumerate}
		
	\section{Przykłady}
	
	\begin{table}[h]
		\centering
		\begin{tabular}{|c|c|c|c|c|}
			\hline
			$i$ & $c$ & $s$ & Jacobi & WartoscDokladna \\
			\hline
			1 & 0.01 & 0.99995 & Inf & -0.85037 \\
			2 & 0.05125 & 0.99869 & Inf & -0.80518 \\
			3 & 0.0925 & 0.99571 & Inf & -0.32141 \\
			4 & 0.13375 & 0.99102 & Inf & -0.79121 \\
			5 & 0.175 & 0.98457 & Inf & -0.32313 \\
			6 & 0.21625 & 0.97634 & Inf & -0.33631 \\
			7 & 0.2575 & 0.96628 & Inf & -0.079279 \\
			8 & 0.29875 & 0.95433 & Inf & -0.68219 \\
			9 & 0.34 & 0.94043 & Inf & -0.85077 \\
			10 & 0.38125 & 0.92447 & Inf & -0.25073 \\
			11 & 0.4225 & 0.90636 & Inf & -0.50496 \\
			12 & 0.46375 & 0.88597 & Inf & -0.64472 \\
			13 & 0.505 & 0.86312 & Inf & -0.41116 \\
			14 & 0.54625 & 0.83762 & -Inf & 0.23923 \\
			15 & 0.5875 & 0.80922 & Inf & -0.16152 \\
			16 & 0.62875 & 0.77761 & Inf & -0.16227 \\
			17 & 0.67 & 0.74236 & -Inf & 0.31244 \\
			18 & 0.71125 & 0.70294 & -0.19163 & -0.19163 \\
			19 & 0.7525 & 0.65859 & 0.3817 & 0.3817 \\
			20 & 0.79375 & 0.60824 & 0.079421 & 0.079421 \\
			21 & 0.835 & 0.55025 & 0.315 & 0.315 \\
			22 & 0.87625 & 0.48186 & 0.45397 & 0.45397 \\
			23 & 0.9175 & 0.39774 & 0.36222 & 0.36222 \\
			24 & 0.95875 & 0.28425 & 0.43201 & 0.43201 \\
			25 & 1 & 0 & 0.56395 & 0.56395 \\
			\hline
		\end{tabular}
		\caption{Zestawienie wektorów $c$, $s$ oraz odpowiadających im wyników funkcji $Jacobi$ oraz rozwiązania operacją wbudowaną. Ze względu na czytelność zaprezentowane zostały pierwsze połowy wyników funkcji $Jacobi$ oraz dokładne wartości (wektory wynikowe są długości 2 razy większej niż długość $c$). Proszę uwierzyć, że wyniki drugiej połowy są analogiczne do pokazanych - dla pewnych początkowych wartości $c$ funckja $Jacobi$ zwraca $\pm \infty$, dla pozostałych dokładne wyniki.}
	\end{table}

	\begin{table}[h]
		\centering
		\begin{tabular}{|c|c|c|c|c|}
			\hline
			$i$ & $c$ & $s$ & Jacobi & WartoscDokladna \\
			\hline
			1 & 0.7 & 0.71414 & Inf & -32.629 \\
			2 & 0.70045 & 0.7137 & -Inf & 13.183 \\
			3 & 0.70091 & 0.71325 & -Inf & 38.106 \\
			4 & 0.70136 & 0.7128 & Inf & -16.099 \\
			5 & 0.70182 & 0.71236 & Inf & -17.501 \\
			6 & 0.70227 & 0.71191 & Inf & -48.676 \\
			7 & 0.70273 & 0.71146 & -Inf & 20.152 \\
			8 & 0.70318 & 0.71101 & Inf & -10.99 \\
			9 & 0.70364 & 0.71056 & -Inf & 12.001 \\
			10 & 0.70409 & 0.71011 & -Inf & 20.345 \\
			11 & 0.70455 & 0.70966 & -Inf & 17.683 \\
			12 & 0.705 & 0.70921 & $4.6574 \times 10^{259}$ & -18.192 \\
			13 & 0.70545 & 0.70876 & $7.4025 \times 10^{203}$ & -14.102 \\
			14 & 0.70591 & 0.7083 & $2.431 \times 10^{148}$ & -24.497 \\
			15 & 0.70636 & 0.70785 & $7.14 \times 10^{92}$ & -41.289 \\
			16 & 0.70682 & 0.7074 & $-5.1039 \times 10^{36}$ & 18.384 \\
			17 & 0.70727 & 0.70694 & 3.9028 & 3.9028 \\
			18 & 0.70773 & 0.70649 & 11.176 & 11.176 \\
			19 & 0.70818 & 0.70603 & 26.714 & 26.714 \\
			20 & 0.70864 & 0.70557 & -29.318 & -29.318 \\
			21 & 0.70909 & 0.70512 & 21.065 & 21.065 \\
			22 & 0.70955 & 0.70466 & 36.876 & 36.876 \\
			23 & 0.71 & 0.7042 & -2.334 & -2.334 \\
			\hline
		\end{tabular}
		\caption{Wartość graniczna zbieżności metody. Dla małych wartości $c_i$ metoda nie jest zbieżna, dla dużych jest. Granicą pomiędzy małymi a dużymi wartościami $c_i$ jest $\frac{\sqrt{2}}{2} \approx 0.707106$. Jeśli każdy element wektora $c$ jest $\ge \frac{\sqrt{2}}{2}$, to macierz $A$ jest diagonalnie dominująca, więc metoda jest zbieżna. Można postawić hipotezę, że jest to również warunek konieczny zbieżności metody.}
	\end{table}

	\begin{table}[h]
		\centering
		\begin{tabular}{|c|c|c|c|}
			\hline
			$i$ & $b$ & Jacobi & WartoscDokladna \\
			\hline
			1 & $0.7066 \times 10^6$ & $6.4945 \times 10^5$ & $6.4945 \times 10^5$ \\
			2 & $7.0461 \times 10^6$ & $4.6581 \times 10^6$ & $4.6581 \times 10^6$ \\
			3 & $6.9846 \times 10^6$ & $5.6726 \times 10^6$ & $5.6726 \times 10^6$ \\
			4 & $6.6057 \times 10^6$ & $3.513 \times 10^6$ & $3.513 \times 10^6$ \\
			5 & $5.9652 \times 10^6$ & $4.7952 \times 10^6$ & $4.7952 \times 10^6$ \\
			6 & $6.1906 \times 10^6$ & $3.72 \times 10^6$ & $3.72 \times 10^6$ \\
			7 & $6.4159 \times 10^6$ & $2.0145 \times 10^6$ & $2.0145 \times 10^6$ \\
			8 & $1.0285 \times 10^6$ & $-1.559 \times 10^6$ & $-1.559 \times 10^6$ \\
			9 & $1.2370 \times 10^6$ & $-1.6386 \times 10^6$ & $-1.6386 \times 10^6$ \\
			10 & $9.8860 \times 10^6$ & $4.187 \times 10^6$ & $4.187 \times 10^6$ \\
			11 & $1.5222 \times 10^6$ & $6.2 \times 10^5$ & $6.2 \times 10^5$ \\
			12 & $4.8879 \times 10^6$ & $4.0508 \times 10^6$ & $4.0508 \times 10^6$ \\
			13 & $7.2602 \times 10^6$ & $4.7304 \times 10^6$ & $4.7304 \times 10^6$ \\
			14 & $8.7414 \times 10^6$ & $7.6701 \times 10^6$ & $7.6701 \times 10^6$ \\
			15 & $1.7486 \times 10^6$ & $1.238 \times 10^6$ & $1.238 \times 10^6$ \\
			16 & $2.7612 \times 10^6$ & $3.38 \times 10^5$ & $3.38 \times 10^5$ \\
			17 & $2.4845 \times 10^6$ & $1.1169 \times 10^6$ & $1.1169 \times 10^6$ \\
			18 & $1.9555 \times 10^6$ & $-1.1115 \times 10^6$ & $-1.1115 \times 10^6$ \\
			19 & $9.6078 \times 10^6$ & $8.3133 \times 10^6$ & $8.3133 \times 10^6$ \\
			20 & $1.0495 \times 10^6$ & $-2.5291 \times 10^6$ & $-2.5291 \times 10^6$ \\
			\hline
		\end{tabular}
		\caption{Zestawienie dla dużych wartości wekotra $b$ dla metody zbieżnej ($0.8 \le c_i \le 1$ dla każdego $i$). Wartości wektora $b$ nie psują zbieżności.}
	\end{table}

	\begin{table}[h]
		\centering
		\begin{tabular}{|c|c|c|}
			\hline
			Rozmiar $c$ & Rozmiar $A$ & Zbieżność \\
			\hline
			$100$ & $200 \times 200$ & TAK \\
			$500$ & $1000 \times 1000$ & TAK \\
			$1000$ & $2000 \times 2000$ & TAK\\
			\hline
		\end{tabular}
		\caption{Zbadaliśmy również wpływ rozmiaru danych na zbieżność metody. Dla wektora $c$ długości $100$, $500$, $1000$ metoda była zbieżna z ograniczeniem z przykładu 2. Wniosek: Zbieżność metody nie zależy od rozmiaru danych wejściowych. Ze względu na czytelność nie umieszczamy tu tabeli potwierdzających te obserwacje.}
	\end{table}

	\begin{table}[h]
		\centering
		\begin{tabular}{|c|c|c|c|c|c|c|c|c|c|c|c|}
			\hline
			Próba & 1 & 2 & 3 & 4 & 5 & 6 & 7 & 8 & 9 & 10 & Średnia \\
			Rozmiar $A$ & & & & & & & & & & & \\
			\hline
			$10 \times 10$ & $67$ & $59$ & $51$ & $74$ & $54$ & $68$ & $59$ & $57$ & $66$ & $73$ & $62.8$ \\
			$20 \times 20$ & $80$ & $79$ & $75$ & $80$ & $78$ & $75$ & $77$ & $67$ & $80$ & $81$ & $77.2$ \\
			$40 \times 40$ & $78$ & $73$ & $82$ & $80$ & $78$ & $77$ & $69$ & $74$ & $80$ & $83$ & $77.4$ \\
			$100 \times 100$ & $77$ & $83$ & $81$ & $82$ & $80$ & $80$ & $79$ & $80$ & $81$ & $78$ & $80.1$ \\
			$200 \times 200$ & $79$ & $82$ & $82$ & $84$ & $81$ & $84$ & $81$ & $81$ & $82$ & $82$ & $81.8$ \\
			$1000 \times 1000$ & $85$ & $86$ & $85$ & $85$ & $86$ & $85$ & $85$ & $85$ & $85$ & $86$ & $85.3$ \\
			\hline
		\end{tabular}
		\caption{Liczba iteracji w zależności od rozmiaru macierzy $A$. Analizę przeprowadzono dla wartości $c_i \ge 0.8$. Wraz ze zwiększaniem rozmiaru danych liczba iteracji potrzebna do uzyskania wyniku nieznacznie rośnie.}
	\end{table}

	\begin{table}[h]
		\centering
		\begin{tabular}{|l|*{10}{c}|c|}
			\hline
			Próba & 1 & 2 & 3 & 4 & 5 & 6 & 7 & 8 & 9 & 10 & Średnia \\
			Zakres $c_i$ & & & & & & & & & & & \\
			\hline
			$[0.95, 1]$ & $22$ & $22$ & $22$ & $22$ & $22$ & $22$ & $22$ & $22$ & $22$ & $22$ & $22$ \\
			$[0.90, 1]$ & $33$ & $33$ & $33$ & $34$ & $33$ & $32$ & $33$ & $33$ & $34$ & $33$ & $33.1$ \\
			$[0.85, 1]$ & $49$ & $51$ & $49$ & $49$ & $51$ & $49$ & $46$ & $50$ & $48$ & $43$ & $48.5$ \\
			$[0.80, 1]$ & $81$ & $78$ & $78$ & $77$ & $82$ & $79$ & $81$ & $79$ & $80$ & $80$ & $79.5$ \\
			\hline
		\end{tabular}
		\caption{Liczba iteracji potrzebnych do otrzymania wyniku w zależności od zakresu wartości elementów wektora $c$. Zestawienie przygotowano dla macierzy $A$ rozmiaru $100 \times 100$ ($c$ ma tu długość $50$). Im wartości $c_i$ są bliższe $1$ tym szybsza jest metoda.}
	\end{table}

	\clearpage
	
	\section{Zastosowania}
	\begin{enumerate}
		\item Iteracyjny algorytm znajdowania wartości własnych macierzy (Eigenvalue Problem)
		\item Jądrowe maszyny wektorów nośnych (Kernelized Support Vector Machines)
	\end{enumerate}
	
	\section*{Źródła}
	\begin{enumerate}
		\item Kicaid A. "Analiza numeryczna"
		\item Notatki do wykładu Wróbel I. "Metody numeryczne", MiNI 2023, semestr zimowy
		\item https://wazniak.mimuw.edu.pl/index.php?title=MN08
	\end{enumerate}
	
\end{document}
